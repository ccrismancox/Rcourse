
\PassOptionsToPackage{obeyspaces, hyphens}{url}
%\usepackage[top=1.00in,bottom=1.00in,left=1in,right=1in]{geometry}
\usepackage{amsmath, amsfonts,amssymb, amsthm, setspace, fancyhdr, url, array, float, graphicx, color, verbatim, pdflscape}
\usepackage{afterpage}
\usepackage[linesnumbered,ruled,vlined]{algorithm2e}
\usepackage{longtable}
\usepackage[longnamesfirst, sort]{natbib}
\usepackage{rotating, booktabs}
\usepackage[labelfont=bf]{caption,subcaption}
%\usepackage[colorlinks=true,anchorcolor=red,citecolor=blue,linkcolor=red,breaklinks=true]{hyperref}
% \usepackage{mathpazo}
\usepackage{listings}
\usepackage{ragged2e}
\usepackage{enumitem}

\usepackage{dcolumn}

\newcolumntype{d}{D{.}{.}{-1}}

\newcolumntype{,}{D{,}{,\,}{-1}}

\newcommand*{\myalign}[2]{\multicolumn{1}{#1}{#2}}

\newcommand\independent{\protect\mathpalette{\protect\independenT}{\perp}}
\def\independenT#1#2{\mathrel{\rlap{$#1#2$}\mkern2mu{#1#2}}}

%\pagestyle{fancy}
\addtocounter{page}{-1}
\onehalfspacing

\DeclareMathOperator*{\argmin}{\operatornamewithlimits{argmin}}
\DeclareMathOperator*{\argmax}{\operatornamewithlimits{argmax}}
\DeclareMathOperator{\E}{E}
\DeclareMathOperator{\Var}{Var}
\DeclareMathOperator{\Cov}{Cov}
\newcommand{\coxnote}[1]{%
	\marginpar{{
			\parbox{\marginparwidth}{%
				\setstretch{0.5}{\textcolor{blue}{\scriptsize{#1}}}\raggedright
			}
	}}
}
\newcommand{\R}{\textsf{R}}
\newtheoremstyle{myStyle}
{0pt} % Space above
{7pt} % Space below
{\itshape} % Body font
{} % Indent amount
{\bfseries} % Theorem head font
{} % Punctuation after theorem head
{.5em} % Space after theorem head
{} % Theorem head spec (can be left empty, meaning `normal')

\theoremstyle{myStyle}
\newtheorem{theorem}{Theorem}
\newtheorem{proposition}{Proposition}
\newtheorem{fact}{Fact}
\newtheorem{acknowledgement}{Acknowledgement}
\newtheorem{assumption}{Assumption}
\newtheorem{claim}{Claim}
\newtheorem{conclusion}{Conclusion}
\newtheorem{condition}{Condition}
\newtheorem{conjecture}{Conjecture}
\newtheorem{corollary}{Corollary}
\newtheorem{atheorem}{Theorem}
\newtheorem{result}{Result}
\newtheorem{definition}{Definition}
\newtheorem{property}{Property}
\newtheorem{example}{Example}
\newtheorem{lemma}{Lemma}
\newtheorem{remark}{Remark}
\newtheorem*{note}{Note}

\newtheorem{assumptionA}{Assumption \ignorespaces}
\renewcommand{\theassumptionA}{A\arabic{assumptionA}}
\newtheorem{propertyA}{Property \ignorespaces}
\renewcommand{\thepropertyA}{A\arabic{propertyA}}

\newtheorem{assumptionB}{Assumption \ignorespaces}
\renewcommand{\theassumptionB}{B\arabic{assumptionB}}
\newtheorem{propertyB}{Property \ignorespaces}
\renewcommand{\thepropertyB}{B\arabic{propertyB}}

\newtheorem{assumptionC}{Assumption \ignorespaces}
\renewcommand{\theassumptionC}{C\arabic{assumptionC}}
\newtheorem{propertyC}{Property \ignorespaces}
\renewcommand{\thepropertyC}{C\arabic{propertyC}}

\newtheorem{assumptionD}{Assumption \ignorespaces}
\renewcommand{\theassumptionD}{D\arabic{assumptionD}}
\newtheorem{propertyD}{Property \ignorespaces}
\renewcommand{\thepropertyD}{D\arabic{propertyD}}

\newtheorem{assumptionE}{Assumption \ignorespaces}
\renewcommand{\theassumptionE}{E\arabic{assumptionE}}
\newtheorem{propertyE}{Property \ignorespaces}
\renewcommand{\thepropertyE}{E\arabic{propertyE}}


\usepackage{tikz}
\usetikzlibrary{patterns,arrows,decorations.pathreplacing}
\usepackage{xcolor}
\usepackage{colortbl} 
\usepackage{multirow}
\usepackage{multicol}
\usepackage{graphicx}
 \usepackage{diagbox} 
